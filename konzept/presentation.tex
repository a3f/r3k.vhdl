%% LaTeX-Beamer template for KIT design
%% by Erik Burger, Christian Hammer
%% edited by Cihat Gündüz, Peter Wolf
%%
%% version 3.0
%%
%% mostly compatible to KIT corporate design v2.0
%% http://intranet.kit.edu/gestaltungsrichtlinien.php
%%
%% Problems, bugs and comments to
%% burger@kit.edu

%% Modified by Sebastian Friebe and Johannes Bechberger

% Selbstdefinierte Kommandos als Arbeitsbeschleunigung:
\newcommand{\f}[1]{\textbf{#1}}					% nutzen in der Form: \f{FETTER TEXT}
\newcommand{\p}{\pause}						% nutzen in der Form: \p (statt \pause}
%\newcommand{\code}[1]{\lstinputlisting[caption=#1]}		% Nutzen in der Form: \code{TITEL}{QUELLE}
\newcommand{\code}[1]{\lstinputlisting[title=#1]}		% Nutzen in der Form: \code{TITEL}{QUELLE}
\newcommand{\cd}[1]{\lstinline[basicstyle=\normalsize\ttfamily]{#1}}		% Nutzen in der Form: \cd{KURZER QUELLTEXT}
\newcommand{\scd}[1]{\lstinline[basicstyle=\scriptsize\ttfamily]{#1}}	% Nutzen in Form: \scd{KLEINER QUELLTEXT} 	% scd steht für Small CoDe
\newcommand{\mcd}[1]{\lstinline[basicstyle=\footnotesize\ttfamily]{#1}}	% für mittlere Größe verwenden!
\newcommand{\bcode}[1]{\lstinputlisting[title=#1,basicstyle=\normalsize\ttfamily]}	% Big Code, Nutzen in der Form: \bcode{TITEL}{QUELLE}
\newcommand{\scode}[1]{\lstinputlisting[title=#1,basicstyle=\tiny\ttfamily]} 	%Small Code, Nutzen in der Form: \scode{TITEL}{QUELLE}

%\newcommand{\trash_to_remove_orange_highlighting}{\end{verbatim}}

\newcommand{\task}[3]{
	\subsection{Aufgabe #1}
	\begin{frame}
		\frametitle{Aufgabe #1}
		#2
		\invisible<1> {
			#3
		}
	\end{frame}

	\begin{frame}
		\frametitle{Aufgabe #1}
		#2
		#3
	\end{frame}
}

\newenvironment{descr}{%
	\newcommand\itemz[2][]{\item[\textbf{##1}] ##2}%
	\begin{description}}{\end{description}%
}

\newenvironment{myitemize}{%
	\newcommand{\conseq}[2][]{\item[\color{kit-green100}\textbf{$\Rightarrow$}] ##2}
	\begin{itemize}}{\end{itemize}%
}

\newcommand{\taskdescr}[3]{
	\task{#1}{
		#2 \\ \vspace{0.2cm}
	}{
		\begin{descr}
		#3
		\end{descr}
	}
}

\newcommand{\taskitemize}[3]{
	\task{#1}{#2}{
	\begin{myitemize}
		#3
	\end{myitemize}
}
}

%% SLIDE FORMAT

\documentclass[18pt]{beamer}

% use 'beamerthemekit' for standard 4:3 ratio
% for widescreen slides (16:9), use 'beamerthemekitwide

\usepackage{templates/beamerthemekit}
% \usepackage{templates/beamerthemekitwide}

% Erlaube Code-Integration mit listings
\definecolor{kit-gray}{RGB}{224,224,224}
\definecolor{kit-green}{RGB}{32,149,128}
\usepackage{listings}
\usepackage{courier}
\usepackage{animate}

\lstset{
         language=C,
%         basicstyle=\scriptsize\ttfamily, % Skriptgröße und Standardschrift
		 basicstyle=\tiny,
         numbers=left,              	% Ort der Zeilennummern
         numberstyle=\tiny,         	% Stil der Zeilennummern
         %stepnumber=2,               	% Abstand zwischen den Zeilennummern
         numbersep=5pt,              	% Abstand der Nummern zum Text
         tabsize=2,                  		% Größe von Tabs
         extendedchars=true,         %
         breaklines=true,            	% Zeilen werden umgebrochen
         %keywordstyle=\color{red},
    	frame=t,         
	%frameround=tftf, 
         keywordstyle=[1]\textbf,    	% Stil der Keywords
         stringstyle=\color{blue}\ttfamily, 	% Farbe der String
         showspaces=false,           	% Leerzeichen anzeigen?
         showtabs=false,             	% Tabs anzeigen?
         showlines=true,              % Leerzeilen am Ende?
         xleftmargin=17pt,
         framexleftmargin=17pt,
         framexrightmargin=6pt,
         framexbottommargin=4pt,
         backgroundcolor=\color{kit-gray},
         commentstyle=\color{kit-green},
         showstringspaces=true    	% Leerzeichen in Strings anzeigen?       
         %numberbychapter=false 
 }

\usepackage{caption}
\DeclareCaptionFont{white}{\color{white}}
\DeclareCaptionFormat{listing}{\colorbox[cmyk]{0.79, 0.18, 0.57,0.03}{\parbox{\textwidth}{\hspace{4pt}#1#2#3}}}
% Taken out since it creats a warning (and I have no idea what it is for)
%\captionsetup[lstlisting]{format=listing,labelfont=white,textfont=white, singlelinecheck=false, margin=0pt, font={bf,footnotesize}}


%% TITLE PICTURE

% if a custom picture is to be used on the title page, copy it into the 'logos'
% directory, in the line below, replace 'mypicture' with the 
% filename (without extension) and uncomment the following line
% (picture proportions: 63 : 20 for standard, 169 : 40 for wide
% *.eps format if you use latex+dvips+ps2pdf, 
% *.jpg/*.png/*.pdf if you use pdflatex)

\titleimage{kit_title}

%% TITLE LOGO

% for a custom logo on the front page, copy your file into the 'logos'
% directory, insert the filename in the line below and uncomment it

\titlelogo{nogo}

% (*.eps format if you use latex+dvips+ps2pdf,
% *.jpg/*.png/*.pdf if you use pdflatex)


% DEUTSCHE SPRACHE EINBINDEN
\usepackage[utf8]{inputenc}

% Deutsche Ausgabe anpassen
\usepackage[T1]{fontenc}

% Für korrekte Unterstreichung etc.
%\usepackage[normalem]{ulem}

% Zeilentrennung
\usepackage[ngerman]{babel}

% disable all navigation symbols
%\beamertemplatenavigationsymbolsempty
% disable only the next-slide, etc, symbols
\setbeamertemplate{navigation symbols}{}

%% TikZ INTEGRATION

% use these packages for PCM symbols and UML classes
% \usepackage{templates/tikzkit}
% \usepackage{templates/tikzuml}

% Only sections in the \tableofcontents
\setcounter{tocdepth}{1}

% Default values
\institute{Lehrstuhl Systemarchitektur}
\title[r3k.vhdl - MIPS R3000 auf dem FPGA]{r3k.vhdl - MIPS R3000 auf einem FPGA}
\author{Aicha Ben Chaouacha, Ahmad Fatoum, Niklas Fuhrberg}


\newcommand\myheading[1]{%
  \par\bigskip
  {\Large\bfseries#1}\par\smallskip}
\setbeamertemplate{caption}{\raggedright\insertcaption\par}
\newcommand{\ebackupbegin}{
   \newcounter{finalframe}
   \setcounter{finalframe}{\value{framenumber}}
}
\newcommand{\ebackupend}{
   \setcounter{framenumber}{\value{finalframe}}
}

\usepackage{color}
\usepackage{listings}
\usepackage{eurosym}
\usepackage{stfloats}
\usepackage{subfig}

\definecolor{javared}{rgb}{0.6,0,0} % for strings
\definecolor{javagreen}{rgb}{0.25,0.5,0.35} % comments
\definecolor{javapurple}{rgb}{0.5,0,0.35} % keywords
\definecolor{javadocblue}{rgb}{0.25,0.35,0.75} % javadoc

\lstset{language=C,
	basicstyle=\tiny\ttfamily,
	keywordstyle=\color{javapurple}\bfseries,
	stringstyle=\color{javared},
	commentstyle=\color{javagreen},
	morecomment=[s][\color{javadocblue}]{/**}{*/},
	numbers=none,
	numberstyle=\tiny\color{black},
	stepnumber=1,
	numbersep=0pt,
	tabsize=2,
	showspaces=false,
	showstringspaces=false}

% unten in der preamble.tex sind noch weitere default-werte

\subtitle{Basispraktikum Technische Informatik}
\date{05.06.2017}

% Um das blaue Titel-Bild zu entfernen
%\titleimage{nogo}

\begin{document}

\begin{frame}
	\titlepage
\end{frame}


\section{MIPS I ISA}

\begin{frame}{Der MIPS R3000}
\begin{itemize}
       \item 1988 auf dem Markt gekommen
       \item 32-bit, RISC Architektur
		\item 32 Register
		\item 5-Stufige Pipeline
\end{itemize}

\begin{center}
\includegraphics[scale=0.22]{R3000.jpg}
\end{center}

\end{frame}


\begin{frame}{Was wird weg gelassen ?}
\begin{itemize}
       \item Traps
       \item Interrupts 
		\item Overflow
		\item Syscall
        \item Betriebsmodi (Kernel/User)
 		\item MMU
 		\item Caching

\end{itemize}
\end {frame}
\begin{frame}{MIPS Befehlssatz}
\begin{center}
\includegraphics[keepaspectratio=true,width=0.85\paperwidth]{ISA.png}
\end{center}

\end{frame}

\begin{frame}{MIPS Befehlsformate}
%% some quote about real-time systems
\begin{itemize}
		\item R-Type -- Arithmetik and Logik
       \item I-Type -- Laden/Speichern, Verzweigen, Direktwert
       \item J-Format -- Jump
\end{itemize}

\begin{center}
\includegraphics[keepaspectratio=true,width=0.83\paperwidth]{instruction-formats.jpeg}
\end{center}
\end{frame}
\begin{frame}{MIPS Pipeline}
%% leakage currents
\begin{center}

\includegraphics[keepaspectratio=true,width=0.83\paperwidth]{pipeline.png}
\end{center}


\end{frame}

\begin{frame}{Wie werden wir das Ganze implementieren? Bsp.: ALU}

%% leakage currents
\begin{center}

\includegraphics[keepaspectratio=true,width=.4\linewidth]{Alu.jpeg}
\includegraphics[keepaspectratio=true,width=.7\linewidth]{Alu-code.jpeg}
\end{center}
\end{frame}
\section{Unser MIPS System}

\begin{frame}{Hardware-Komponenten}

\begin{itemize}
	\item MIPS R3000
	\item 32bit
\end{itemize}

\begin{center}
\includegraphics[scale=0.40]{bus.jpeg} %%Grafik s. WA
\end{center}

\end{frame}

\begin{frame}{UART}

\begin{itemize}
	\item Universal Asynchronous Receiver Transmitter
	\item NS8250 Class 
	\item Adresse: 0x1fd00x3f8
	\item Eingabe
	\item Priorität -> Erlaubt debugging
	\item keine interrupts
\end{itemize}
\begin{center}
%%In deiner voice message klingst als hättest du ne schöne grafik. Falls ja > hier rein. Sonst erklär ichs bloß
\end{center}


\end{frame}

\begin{frame}{VGA Frame Puffer}
	\begin{itemize}
		\item Adresse: 0x1fd0????
		\item C Programm schreibt schreibt in Array
		\item Array wird aus Speicher gelesen
		\item VGA Clock
	\end{itemize}
\end{frame}

\begin{frame}{VGA}
	\begin{itemize}		
		\item 255 Farben
		\item Größe: 160*120 -> 19k byte
		\item klein anfangen, feststellen ob der Speicher reicht
	\end{itemize}
\end{frame}


\section{Wie kommt die Software auf den FPGA?}

\begin{frame}{Strom da. Was nun?}

Reset Vektor wird angesprungen. Assembler Code verantwortlich für

\begin{itemize}
\item Globale Variablen (.BSS) nullen
\item Daten von ROM in den RAM kopieren
\item Stack initialisieren
\item main() im C Code anspringen
\end{itemize}

\end{frame}

\begin{frame}{BIOS (Basic Input/Output System)}

Verantwortlich für

\begin{itemize}
\item Initialisiert UART
\item Startet Kommandozeilen Interpreter auf UART
\item Startet Applikationen
\item Interrupts und Traps?
\end{itemize}

\end{frame}

\begin{frame}{C-Toolchain}

\begin{itemize}
\item GCC 6.2.0 MIPS Barebones Cross-Compiler
\item \texttt{-ffreestanding -nostdlib -nostartfiles}
\item Wie kommt das Programm in den Speicher?
\begin{itemize}
    \item Linker Script: Spezifiziert Addressvergabe an Symbole
    \item objcopy: Extrahiert rohen Maschinen Code
    \item iMPACT: Spielt extrahierte Binärdatei auf ROM
\end{itemize}
\end{itemize}
\begin{center}
\includegraphics[scale=0.10]{gcc.png}
\end{center}


\end{frame}

\begin{frame}{Komplexität im Griff behalten?}

\begin{itemize}
\item Emulation:
\begin{itemize}
    \item QEMU mit R3000, UART und VGA
    \item Code lokal Testebar vor Auspielen auf den FPGA
\end{itemize}
    \item Simulation: Testbenches: z.B. ALU und Instruktionsdecoder. Alleinstehend, gut testbar.
\end{itemize}
\begin{center}
\includegraphics[scale=0.85]{qemu.png}
\end{center}



\end{frame}


\begin{frame}{Applikationen}
\begin{itemize}
\item Mandebrot-Fraktal
\item Vllt. DOS Demos portieren
\end{itemize}

\begin{center}
\includegraphics[keepaspectratio=true,width=0.40\paperwidth]{mandelbrot.png}
\end{center}


\end{frame}


\begin{frame}{Doom?}
\begin{center}
\includegraphics[keepaspectratio=true,width=0.40\paperwidth]{doom.png}
\end{center}
 

\end{frame}

\begin{frame}{Aufgaben}
\begin{itemize}
\item ALU (Aicha)
\item Instruction Decoder (Ahmad)
\item UART (Niklas)
\item Address decoder
\item ROM
\item RAM
\item VGA frame buffer
\item Register File
\item Pipeline (Data path)
\end{itemize}	
\end{frame}

\begin{frame}{Fragen? Anregungen?} %%Solche Folien macht man nicht. Besser wäre Zusammenfassung.
\begin{center}
\includegraphics[keepaspectratio=true,width=0.40\paperwidth]{doom.png}
\end{center}
 


\end{frame}

\end{document}
