%% LaTeX-Beamer template for KIT design
%% by Erik Burger, Christian Hammer
%% edited by Cihat Gündüz, Peter Wolf
%%
%% version 3.0
%%
%% mostly compatible to KIT corporate design v2.0
%% http://intranet.kit.edu/gestaltungsrichtlinien.php
%%
%% Problems, bugs and comments to
%% burger@kit.edu

%% Modified by Sebastian Friebe and Johannes Bechberger

% Selbstdefinierte Kommandos als Arbeitsbeschleunigung:
\newcommand{\f}[1]{\textbf{#1}}					% nutzen in der Form: \f{FETTER TEXT}
\newcommand{\p}{\pause}						% nutzen in der Form: \p (statt \pause}
%\newcommand{\code}[1]{\lstinputlisting[caption=#1]}		% Nutzen in der Form: \code{TITEL}{QUELLE}
\newcommand{\code}[1]{\lstinputlisting[title=#1]}		% Nutzen in der Form: \code{TITEL}{QUELLE}
\newcommand{\cd}[1]{\lstinline[basicstyle=\normalsize\ttfamily]{#1}}		% Nutzen in der Form: \cd{KURZER QUELLTEXT}
\newcommand{\scd}[1]{\lstinline[basicstyle=\scriptsize\ttfamily]{#1}}	% Nutzen in Form: \scd{KLEINER QUELLTEXT} 	% scd steht für Small CoDe
\newcommand{\mcd}[1]{\lstinline[basicstyle=\footnotesize\ttfamily]{#1}}	% für mittlere Größe verwenden!
\newcommand{\bcode}[1]{\lstinputlisting[title=#1,basicstyle=\normalsize\ttfamily]}	% Big Code, Nutzen in der Form: \bcode{TITEL}{QUELLE}
\newcommand{\scode}[1]{\lstinputlisting[title=#1,basicstyle=\tiny\ttfamily]} 	%Small Code, Nutzen in der Form: \scode{TITEL}{QUELLE}

%\newcommand{\trash_to_remove_orange_highlighting}{\end{verbatim}}

\newcommand{\task}[3]{
	\subsection{Aufgabe #1}
	\begin{frame}
		\frametitle{Aufgabe #1}
		#2
		\invisible<1> {
			#3
		}
	\end{frame}

	\begin{frame}
		\frametitle{Aufgabe #1}
		#2
		#3
	\end{frame}
}

\newenvironment{descr}{%
	\newcommand\itemz[2][]{\item[\textbf{##1}] ##2}%
	\begin{description}}{\end{description}%
}

\newenvironment{myitemize}{%
	\newcommand{\conseq}[2][]{\item[\color{kit-green100}\textbf{$\Rightarrow$}] ##2}
	\begin{itemize}}{\end{itemize}%
}

\newcommand{\taskdescr}[3]{
	\task{#1}{
		#2 \\ \vspace{0.2cm}
	}{
		\begin{descr}
		#3
		\end{descr}
	}
}

\newcommand{\taskitemize}[3]{
	\task{#1}{#2}{
	\begin{myitemize}
		#3
	\end{myitemize}
}
}

%% SLIDE FORMAT

\documentclass[18pt]{beamer}

% use 'beamerthemekit' for standard 4:3 ratio
% for widescreen slides (16:9), use 'beamerthemekitwide

\usepackage{templates/beamerthemekit}
% \usepackage{templates/beamerthemekitwide}

% Erlaube Code-Integration mit listings
\definecolor{kit-gray}{RGB}{224,224,224}
\definecolor{kit-green}{RGB}{32,149,128}
\usepackage{listings}
\usepackage{courier}
\usepackage{animate}

\lstset{
         language=C,
%         basicstyle=\scriptsize\ttfamily, % Skriptgröße und Standardschrift
		 basicstyle=\tiny,
         numbers=left,              	% Ort der Zeilennummern
         numberstyle=\tiny,         	% Stil der Zeilennummern
         %stepnumber=2,               	% Abstand zwischen den Zeilennummern
         numbersep=5pt,              	% Abstand der Nummern zum Text
         tabsize=2,                  		% Größe von Tabs
         extendedchars=true,         %
         breaklines=true,            	% Zeilen werden umgebrochen
         %keywordstyle=\color{red},
    	frame=t,         
	%frameround=tftf, 
         keywordstyle=[1]\textbf,    	% Stil der Keywords
         stringstyle=\color{blue}\ttfamily, 	% Farbe der String
         showspaces=false,           	% Leerzeichen anzeigen?
         showtabs=false,             	% Tabs anzeigen?
         showlines=true,              % Leerzeilen am Ende?
         xleftmargin=17pt,
         framexleftmargin=17pt,
         framexrightmargin=6pt,
         framexbottommargin=4pt,
         backgroundcolor=\color{kit-gray},
         commentstyle=\color{kit-green},
         showstringspaces=true    	% Leerzeichen in Strings anzeigen?       
         %numberbychapter=false 
 }

\usepackage{caption}
\DeclareCaptionFont{white}{\color{white}}
\DeclareCaptionFormat{listing}{\colorbox[cmyk]{0.79, 0.18, 0.57,0.03}{\parbox{\textwidth}{\hspace{4pt}#1#2#3}}}
% Taken out since it creats a warning (and I have no idea what it is for)
%\captionsetup[lstlisting]{format=listing,labelfont=white,textfont=white, singlelinecheck=false, margin=0pt, font={bf,footnotesize}}


%% TITLE PICTURE

% if a custom picture is to be used on the title page, copy it into the 'logos'
% directory, in the line below, replace 'mypicture' with the 
% filename (without extension) and uncomment the following line
% (picture proportions: 63 : 20 for standard, 169 : 40 for wide
% *.eps format if you use latex+dvips+ps2pdf, 
% *.jpg/*.png/*.pdf if you use pdflatex)

\titleimage{kit_title}

%% TITLE LOGO

% for a custom logo on the front page, copy your file into the 'logos'
% directory, insert the filename in the line below and uncomment it

\titlelogo{nogo}

% (*.eps format if you use latex+dvips+ps2pdf,
% *.jpg/*.png/*.pdf if you use pdflatex)


% DEUTSCHE SPRACHE EINBINDEN
\usepackage[utf8]{inputenc}

% Deutsche Ausgabe anpassen
\usepackage[T1]{fontenc}

% Für korrekte Unterstreichung etc.
%\usepackage[normalem]{ulem}

% Zeilentrennung
\usepackage[ngerman]{babel}

% disable all navigation symbols
%\beamertemplatenavigationsymbolsempty
% disable only the next-slide, etc, symbols
\setbeamertemplate{navigation symbols}{}

%% TikZ INTEGRATION

% use these packages for PCM symbols and UML classes
% \usepackage{templates/tikzkit}
% \usepackage{templates/tikzuml}

% Only sections in the \tableofcontents
\setcounter{tocdepth}{1}

% Default values
\institute{Lehrstuhl für Rechnerarchitektur und Parallelverarbeitung}
\title[r3k.vhdl]{r3k.vhdl - MIPS R3000 auf einem FPGA}
\author{Niklas Fuhrberg, Ahmad Fatoum}
